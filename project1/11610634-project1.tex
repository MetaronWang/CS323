\documentclass{article}
\usepackage[a4paper,scale=0.8,hcentering,bindingoffset=8mm]{geometry} % A4纸大小,缩放80%,设置奇数页右边留空多一点
\usepackage{hyperref}      % 超链接
\usepackage{listings}      % 代码块
\usepackage{courier}       % 字体
\usepackage{fontspec}      % 字体
\usepackage{fancyhdr}      % 页眉页脚相关宏包
\usepackage{lastpage}      % 引用最后一页
\usepackage{amsmath,amsthm,amsfonts,amssymb,bm} %数学
\usepackage{graphicx}      % 图片
\usepackage{subcaption}    % 图片描述
\usepackage{longtable,booktabs} % 表格
\usepackage{ctex}
\usepackage{soul}
\lstset{                  %设置代码块
         basicstyle=\footnotesize\ttfamily,% 基本风格
         numbers=left,    % 行号
         numbersep=10pt,  % 行号间隔 
         tabsize=4,       % 缩进
         extendedchars=true, % 扩展符号?
         breaklines=true, % 自动换行
         language=C,
         frame=leftline,  % 框架左边竖线
         xleftmargin=5pt,% 竖线左边间距
         showspaces=false,% 空格字符加下划线
         showstringspaces=false,% 字符串中的空格加下划线
         showtabs=false,  % 字符串中的tab加下划线
 }
\pagestyle{fancy}         % 页眉页脚风格
\fancyhf{}                % 清空当前设置
\fancyfoot[C]{\thepage\ / \pageref{LastPage}}%页脚中间显示 当前页 / 总页数,把\label{LastPage}放在最后
\begin{document} 
    \begin{titlepage}       % 封面
        \centering
        \includegraphics[width=\textwidth]{../SUSTC_LOGO.png}
        % \vspace*{\baselineskip}
        \rule{\textwidth}{1.6pt}\vspace*{-\baselineskip}\vspace*{2pt}
        \rule{\textwidth}{0.4pt}\\[\baselineskip]
        {\LARGE COMPILIER @Liu Yepang 2019\\[\baselineskip]\small for SUSTech CSE}
        \\[0.2\baselineskip]
        \rule{\textwidth}{0.4pt}\vspace*{-\baselineskip}\vspace{3.2pt}
        \rule{\textwidth}{1.6pt}\\[\baselineskip]
        \scshape
        \vspace*{\baselineskip}
        {\Large Project 1\par }
        Edited by \\[\baselineskip] {汪至圆\par}
        {\Large 11610634\par }
        \vfill
        {\scshape 2019} \\{\large SHENZHEN}\par
    \end{titlepage}

    \section{Lab Demand}
        In this project, I'm required to implement a SPL parser, specifically, your parser
        will perform lexical analysis and syntax analysis on the SPL source code. The parser was
        written by C++/Flex/Bison, and it can output a syntax tree and find some easy grammatical 
        errors for a SPL code.

    \section{Environment}
        The environment for my project is g++ 9.1, flex 2.6, bison 3.4. All the work of coding
        and test was finished in the Manjaro 18.1 which based on Linux 4.19.
    
    \section{Lexical Analyzer Generator}
        The lexical analyzer generator I used is GNU Flex, which is the open source version for the
        lex. In the lex, I define some token following the guide provided by the TA. And I also define some
        token by myself to satisfy the requirements of the grammar analyzer. The token I defined are 
        LEXERR, which will produce the error lexeme and throw an type A error to tell the user there is an unknown
        lexeme. And the comments was also produced in the lex file, I use regular expressions to filter the
        comment lines and comment blocks. I also can find the nested comment block by the regular expression, which is not allowed in the SPL.
        For the token which I define by myself, the code of them is:
        \begin{lstlisting}
A [/]
B [*]
C [^*/]
comment "/*"{A}*({C}{A}*|{B}|{C})*"*/"
%%
\/\/.*                  {comment line}
"/*"{A}*({C}{A}*|{B}|{C})*{comment}(.|\n)*"*/"  {nested comment block}
{comment}               {comment block}
0x[0-9A-Za-z]+          {yylval=createNode(yytext); return LEXERR; }
('\\x[0-9a-zA-Z]*')     {yylval=createNode(yytext); return LEXERR; }
[0-9][a-zA-Z_0-9]*      {yylval=createNode(yytext); return LEXERR; }
.                       {yylval=createNode(yytext); return LEXERR;}
%%
        \end{lstlisting}
    \section{Parser Generator}
        The parser generator I used is GNU Bison. In the bison, I define the syntax rule following the guide and add some syntax rule
        by myself to catch the error and output the error message. In the bison I define a struct called node, which is used to record the
        message of the syntax tree, it has a member variable with type string to store the node name and line number, it also have a member
        variable with type vector to store the the subnode of it. When process each syntax rule except which using to catch the error, 
        I will get the line number of the code and the add the nodes return by the subrules into the subnode list. If the bison enter a rule which
        use to catch the error, I will add a error message to a list which is used to storage the error. After analysis the code, I will check the size of the
        list which used to store the error, if it is empty, I will output the syntax tree Recursively, or I will output the error messages.
        \subsection{Struct node}
            \begin{lstlisting}
struct Node{
    vector<Node> subNode;
    string show;
};

Node createNode(string s){
    Node node;
    node.show = s;
    return node;
}                       
            \end{lstlisting}
        \subsection{Additional Rules}
            \begin{itemize}
                \item Extdef: 
                \begin{itemize}
                    \item Specifier ExtDecList error 
                    \item Specifier error
                \end{itemize}
                \item FunDec:
                \begin{itemize}
                    \item ID LP error
                    \item ID LP VarList error
                    \item LEXERR LP VarList RP
                    \item LEXERR LP RP
                \end{itemize}
                \item Def:
                \begin{itemize}
                    \item Specifier DecList error
                \end{itemize}
                \item Exp:
                \begin{itemize}
                    \item ID LP Args error
                    \item ID LP error
                    \item LEXERR LP Args error
                    \item Exp LEXERR Exp
                    \item LEXERR LP Args RP
                    \item LEXERR LP RP
                    \item Exp DOT LEXERR 
                    \item LEXERR
                \end{itemize}
            \end{itemize}
    \section{Run}
        My project was compiled by the make. If you want to compile the project, you just need run `\hl{make splc}'. And then it will output a executable file splc.out. 
        The result of running splc.out will print in the stdout of console, if you want to input a spl code file and get an out file, you can use `splc.py' in the bin/, input
        the path of the code file which will be inputted and the result will be outputted to the file with the name replace the `.spl' in the path to the `.out'.
    \section{bonus}
\end{document}