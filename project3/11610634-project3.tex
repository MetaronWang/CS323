\documentclass{article}
\usepackage[a4paper,scale=0.8,hcentering,bindingoffset=8mm]{geometry} % A4纸大小,缩放80%,设置奇数页右边留空多一点
\usepackage{hyperref}      % 超链接
\usepackage{listings}      % 代码块
\usepackage{courier}       % 字体
\usepackage{fontspec}      % 字体
\usepackage{fancyhdr}      % 页眉页脚相关宏包
\usepackage{lastpage}      % 引用最后一页
\usepackage{amsmath,amsthm,amsfonts,amssymb,bm} %数学
\usepackage{graphicx}      % 图片
\usepackage{subcaption}    % 图片描述
\usepackage{longtable,booktabs} % 表格
\usepackage{ctex}
\usepackage{soul}
\usepackage{hyperref}
\lstset{                  %设置代码块
         basicstyle=\footnotesize\ttfamily,% 基本风格
         numbers=left,    % 行号
         numbersep=10pt,  % 行号间隔 
         tabsize=4,       % 缩进
         extendedchars=true, % 扩展符号?
         breaklines=true, % 自动换行
         language=C,
         frame=leftline,  % 框架左边竖线
         xleftmargin=5pt,% 竖线左边间距
         showspaces=false,% 空格字符加下划线
         showstringspaces=false,% 字符串中的空格加下划线
         showtabs=false,  % 字符串中的tab加下划线
 }
\pagestyle{fancy}         % 页眉页脚风格
\fancyhf{}                % 清空当前设置
\fancyfoot[C]{\thepage\ / \pageref{LastPage}}%页脚中间显示 当前页 / 总页数,把\label{LastPage}放在最后
\begin{document} 
    \begin{titlepage}       % 封面
        \centering
        \includegraphics[width=\textwidth]{../SUSTC_LOGO.png}
        % \vspace*{\baselineskip}
        \rule{\textwidth}{1.6pt}\vspace*{-\baselineskip}\vspace*{2pt}
        \rule{\textwidth}{0.4pt}\\[\baselineskip]
        {\LARGE COMPILIER @Liu Yepang 2019\\[\baselineskip]\small for SUSTech CSE}
        \\[0.2\baselineskip]
        \rule{\textwidth}{0.4pt}\vspace*{-\baselineskip}\vspace{3.2pt}
        \rule{\textwidth}{1.6pt}\\[\baselineskip]
        \scshape
        \vspace*{\baselineskip}
        {\Large Project 3\par }
        Edited by \\[\baselineskip] {汪至圆\par}
        {\Large 11610634\par }
        \vfill
        {\scshape 2019} \\{\large SHENZHEN}\par
    \end{titlepage}

    \section{Lab Demand}
        In last two project, I have finished the parser for the SPL source code and generated the syntex 
        tree and do the semantic analysis for the SPL source code. In this project I will generate the intermediate code for my syntex tree.
        

    \section{Environment}
        The environment for my project is g++ 9.1, flex 2.6, bison 3.4, urwid 2.1. All the work of coding
        and test was finished in the Manjaro 18.1 which based on Linux 4.19.
    \section{Run}
        My project was compiled by the make. If you want to compile the project, you just need run `\hl{make splc}'. 
        And then it will output a executable file splc.out. The splc.out will receive a argument which is the path of 
        SPL source code file, and the output of the program will at the same folder of the source code and the extend name of the file 
        will be ".ir". The context of the output file will be the intermediate-code.
    \section{Generate Process}
        When generate the intermediate-code, I use the translation schemes introduced in the document. I define the translation schemes for the expression, the condition, the
        statement, the arguments and the CompSt node. For the assumptions in the document, I traversal the node and find all the definitions of the function and apply the
        translation schemes to it.
        For the translation schemes for CompSt node, I will do the declaration for the variable and add the variable to the variable map which storage the correspondence
        information of the variable name and the name I used in the intermediate-code. And for the Stat node in the CompSt node, I will apply the statement translation scheme
        to it, and in the statement translation scheme, there will use other translation schemes to translate specific node.
    \section{Optimization}
        After generated the intermediate-code, I find there are many repeated statements, so I want to optimize it.
        \begin{itemize}
            \item For all the expression with a const value or a single variable which has been defined, the expression translation scheme won't return the code to set a code 
            to define a new variable but a string with length 0, and it will set the value of the expression be the expression node be the const value or the name used in the 
            intermediate-code of the single variable. And while using this expression node in the right value expression, we can use the value of it directly.
            \item In the intermediate-code, we will use so many temporary variables to transfer the value, and most of them are so redundant. In these step, I will find and
            analysis all the temporary variables. If the temporary which won't be used in the code below it, I will delete the line define the temporary variable. If the 
            temporary variable just used in the right value individually, I will delete the line which used it and set the left value of the line define the temporary variable
            be the left value of the line be deleted.
            \item And I find, there are may be some line which used to define the variable has right value consist of immediate value, for these lines, I will give them the result
            of the expression immediately.
            \item There are some lines have a feature, they have the same left value, and in their right value, there are only the left value of them will appear except the immediate
            value. For these lines, I will compress them to one line.
            \item The last one is a very useful optimization but it can't implement because of the limitation of the simulation. In the simulation, the immediate value can't be used
            in the logical calculation, so we need to define a variable with the value to do the calculation, it's so wasteful to define a variable for each const value in the
            logical calculation. So I want to set a const value pool, all the const values which appear in the logical expressions in the program will be distributed to a variable, and
            when we need to use the const value, we can use the corresponding variable. But this optimization can't be implemented because that the simulation don't support global
            variable. Maybe can create a const value pool for each function, but it is also wasteful.
        \end{itemize}
        \subsection{Performance}
            For the test case privaded by the TA, I test their performance.
            \begin{table}[!htbp]
                \centering
                \label{tab:Performance}
                \begin{tabular}{|c|c|c|}
                    \hline
                    Test Case & Input & Number of Instructions \\
                    \hline
                    test\_3\_r01 &              & 20    \\
                    \hline
                    test\_3\_r02 & 2019,10      & 26    \\
                    \hline
                    test\_3\_r03 &              & 3715  \\
                    \hline
                    test\_3\_r04 & 10           & 18    \\
                    \hline
                    test\_3\_r05 &              & 17    \\
                    \hline
                    test\_3\_r06 &              & 124   \\
                    \hline
                    test\_3\_r07 & 5,10         & 45    \\
                    \hline
                    test\_3\_r08 & -10          & 13    \\
                    \hline
                    test\_3\_r09 &              & 52212 \\
                    \hline
                    test\_3\_r10 & 10           & 95    \\
                    \hline
                \end{tabular}
            \end{table}
    \section{Bonus}
        I haven't implemented bonus features yet.
\end{document}
