\documentclass{article}
\usepackage[a4paper,scale=0.8,hcentering,bindingoffset=8mm]{geometry} % A4纸大小,缩放80%,设置奇数页右边留空多一点
\usepackage{hyperref}      % 超链接
\usepackage{listings}      % 代码块
\usepackage{courier}       % 字体
\usepackage{fontspec}      % 字体
\usepackage{fancyhdr}      % 页眉页脚相关宏包
\usepackage{lastpage}      % 引用最后一页
\usepackage{amsmath,amsthm,amsfonts,amssymb,bm} %数学
\usepackage{graphicx}      % 图片
\usepackage{subcaption}    % 图片描述
\usepackage{longtable,booktabs} % 表格
\usepackage{ctex}
\lstset{                  %设置代码块
         basicstyle=\footnotesize\ttfamily,% 基本风格
         numbers=left,    % 行号
         numbersep=10pt,  % 行号间隔 
         tabsize=4,       % 缩进
         extendedchars=true, % 扩展符号?
         breaklines=true, % 自动换行
         language=C,
         frame=leftline,  % 框架左边竖线
         xleftmargin=5pt,% 竖线左边间距
         showspaces=false,% 空格字符加下划线
         showstringspaces=false,% 字符串中的空格加下划线
         showtabs=false,  % 字符串中的tab加下划线
 }
\pagestyle{fancy}         % 页眉页脚风格
\fancyhf{}                % 清空当前设置
\fancyfoot[C]{\thepage\ / \pageref{LastPage}}%页脚中间显示 当前页 / 总页数,把\label{LastPage}放在最后
\begin{document} 
    \begin{titlepage}       % 封面
        \centering
        \includegraphics[width=\textwidth]{../SUSTC_LOGO.png}
        % \vspace*{\baselineskip}
        \rule{\textwidth}{1.6pt}\vspace*{-\baselineskip}\vspace*{2pt}
        \rule{\textwidth}{0.4pt}\\[\baselineskip]
        {\LARGE COMPILIER @BY 2019\\[\baselineskip]\small for SUSTech CSE}
        \\[0.2\baselineskip]
        \rule{\textwidth}{0.4pt}\vspace*{-\baselineskip}\vspace{3.2pt}
        \rule{\textwidth}{1.6pt}\\[\baselineskip]
        \scshape
        \vspace*{\baselineskip}
        {\Large HomeWork 1\par }
        Edited by \\[\baselineskip] {汪至圆\par}
        {\Large 11610634\par }
        \vfill
        {\scshape 2019} \\{\large SHENZHEN}\par
    \end{titlepage}

    \section{Exercise 1: In a string of length n (n > 0), how many of the following
    are there?}
        \subsection{Prefixes [5 points]}
        There are n+1 prefixes.
        \subsection{Suffixes [5 points]}
        There are n+1 suffixes.
        \subsection{Proper prefixes [5 points]}
        There are n-1 proper prefixes.
        \subsection{Prefixes of length m (0 < m $\leq$ n) [5 points]}
        The number of prefixes of length m is 1.
        \subsection{Suffixes of length m (0 < m $\leq$ n) [5 points]}
        The number of suffixes of length m is 1.
        \subsection{Proper prefixes of length m (0 < m $\leq$ n) [5 points]}
        The number of proper prefixes of length m is 1 if m < n, is 0 if m = n.
        \subsection{Substrings [10 points]}
        There are $\frac{(n+1)n}{2} + 1$ substrings
        \subsection{Subsequences [10 points]}
        \begin{align*}
            \because &\text{For a subsequence of a string. Each char of the string just have two status, in the sequence or not in it.}\\
            \therefore &\text{The number of the status for all the subsequences is: } 2^n\\
            \therefore &\text{∴There are } 2^n \text{ subsequences}
        \end{align*}
    \section{Exercise 2: Describe the languages denoted by the following regular expressions:}
        \subsection{$a(a|b)^*b$ [5 points]}
            The string start with a, there are n a or b behind it(n can be 0), then the string end with b.
            $$s = a + n × (a\quad or\quad b) + b\qquad (n\geq 0)$$
        \subsection{$((\epsilon|a)^*b^*)^*[5 points]$}
            A pattern is start with n a, there are m of b behind it. The string consist of k pattern(m,n,k can be zero).
            $$s = k × (n × (null\quad or\quad a) + m × b)\qquad(m\geq 0, n\geq 0,k\geq 0)$$
        \subsection{$(a|b)^*a(a|b)(a|b) [5 points]$}
            A pattern consist of a or b. The string is start with n pattern, behind it is a, then end with 2 pattern.
            \begin{align*}
                p &= a\quad or\quad b\\
                s &= n × p + a + 2 × p\qquad (n\geq 0)
            \end{align*}
        \subsection{$a^*ba^*ba^*ba^*[5 points]$}
            The pattern consist of n a and b, the string is consist of 3 pattern and b
            \begin{align*}
                p &= n × a + b\\
                s &= 3 × p + m × a\qquad(n\geq 0, m\geq 0)\\
            \end{align*}
    \section{Exercise 3: Write regular definitions for the following languages. Please provide brief explanations why your definitions are correct.}
        \subsection{All strings representing valid telephone numbers in Shenzhen. A valid telephone number
        contains the area code 755 followed by eight digits where the first one cannot be zero (e.g.,
        75588015159). [10 points]}
            \begin{lstlisting}
                755[1−9][0−9]{7}
            \end{lstlisting}
        \subsection{All strings of uppercase letters (A-Z) in which the letters are in ascending lexicographic order. For example, “ABE”is such a string but “BAE”is not (because the letter B appears after A in the alphabet). [10 points]}
            \begin{lstlisting}
                A*B*C*D*E*F*G*H*I*J*K*L*M*N*O*P*Q*R*S*T*U*V*W*X*Y*Z*
            \end{lstlisting}
        \subsection{All strings of lowercase letters that contain the five vowels in order. [10 points]}
            \begin{lstlisting}
                p = [b-df-hj-np-tv−z]
                re = (a|{p})*a(e|{p})*e(i|{p})*i(o|{p})*o(u|{p})*({p})*
            \end{lstlisting}
    \section{Optional Exercises (20 bonus points)}
        \subsection{Exercise 1: Write regular definitions for the following languages. Please provide brief
        explanations why your definitions are correct}
            \subsubsection{Comments consisting of a string surrounded by /* and */, without an intervening */, unless it is
            inside double-quotes ("). [10 points]}
                \begin{lstlisting}
                    /\*(("\*/")|(?!\*/).)*\*/
                \end{lstlisting}
            \subsubsection{All strings of digits with no repeated digits. [10 points]}
                \begin{lstlisting}
                    //If we want to get the longest left substring with no digits from the string:
                    ((?<n>[0-9])(?!\k<n>))*([0-9])
                    //If we want to check the string have no repeat digits(If have repeat digits, we won't get anything)
                    ^((?<n>[0-9])(?!\k<n>))*([0-9])$
                \end{lstlisting}
\end{document}